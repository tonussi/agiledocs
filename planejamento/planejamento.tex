\documentclass[10pt, a4paper]{report}

\usepackage{amsmath}
\usepackage{amsfonts}
\usepackage{amssymb}
\usepackage{graphicx}
\usepackage{listings}
\usepackage{color}

\input{../color.tex}

\lstdefinelanguage{javascript}{ sensitive=true, keywords={ typeof, new, true,
false, catch, function, return, null, catch, switch, var, if, in, while, do,
else, case, break }, otherkeywords={<, >, \/}, ndkeywords={class, export,
boolean, throw, implements, import, this}, comment=[l]{//},
morecomment=[s]{/*}{*/}, morecomment=[s]{<!}{>}, morestring=[b]',
morestring=[b]", alsoletter={-}, alsodigit={:} }

\lstset{%
  backgroundcolor=\color{white},
  basicstyle={\small\ttfamily},
  numbers=right,
  stepnumber=1,
  firstnumber=1,
  numberfirstline=true,
  identifierstyle=\color{black},
  keywordstyle=\color{alizarin}\bfseries,
  ndkeywordstyle=\color{amaranth}\bfseries,
  stringstyle=\color{antiquefuchsia}\ttfamily,
  commentstyle=\color{cinnamon}\ttfamily,
  language={javascript},
  tabsize=2,
  showtabs=false,
  showspaces=false,
  showstringspaces=false,
  extendedchars=true,
  breaklines=true
}

\usepackage[utf8]{inputenc}
\usepackage{fullpage}

\renewcommand\thesection{Questão \arabic{section}}

\title{Questões Sobre Planejamento e Esforço}
\author{Lucas Tonussi - Engenharia de Software}
\date{\today}

\begin{document}

\maketitle
\tableofcontents
\listoftables
\pagebreak

\section{}
\qquad Você foi encarregado de fazer o planejamento de um projeto usando
Processo Unificado. Já foi estimado pela equipe que o projeto (considerado do
início da concepção ao fim da transição) terá de ser desenvolvido em 24 meses
por uma equipe média de 10 pessoas. Porém, o gerente geral da empresa deseja que
você calcule a duração de cada uma das quatro fases do UP bem como o tamanho
recomendado de equipe em cada uma das fases para este projeto. Apresente o
resultado exato com duas casas decimais. Apresente também uma aproximação para
valores inteiros de equipe e duração das fases (em meses) considerando os
seguintes fatores para o projeto:

\subsection{}
\qquad O projeto tem riscos técnicos iniciais significativos que precisam ser
mitigados.

\paragraph{R:}

\textit{Então isso ocasionará aumento do tempo concepção/inception do projeto.
Para mitigação dos riscos. Visão e providência das soluções contra eles.} Pois é
nessa fase que os riscos são identificados, então teremos aqui $C_1, C_2, \dots,
C_n$ para mitigar os riscos técnicos. E nessa alternativa de trabalho, está
sendo exposto que o projeto tem riscos, então eles já foram identificados.
Todavia para cada risco mitigado, ocasiona em trabalho para etapa de elaboração.
Porém, menos que do que se fosse um risco achado em cima da hora.

\begin{center}
  \begin{Large}
    Concepção: $\uparrow$ 15\% do tempo e $\uparrow$ 10\% do esforço.\\
    Elaboração: $\uparrow$ 30\% do tempo e $\uparrow$ 20\% do esforço.
  \end{Large}
\end{center}

\subsection{}
\qquad O novo sistema vai substituir um sistema existente, exigindo testes de
operação, migração de dados e intenso treinamento de usuários.

\paragraph{R:}
Se vamos substituir o sistema existente vai precisar realmente de testes de
migração de dados, onde os dados antigos deverão ser comportados no sistema novo
isso gera mais tempo de transição do projeto. Fase final. Onde varias ciclos
$T_1, T_2, ..., T_n$ serão respeitados para que o sistema seja refinado para
comportar as necessidades do ponto de vista dos dados.

\begin{center}
  \begin{Large}
    Transição: $\uparrow$ 20\% do tempo e $\uparrow$ 20\% do esforço.
  \end{Large}
\end{center}

\subsection{}
\qquad Finalmente, indique, justificando, a duração ideal para as iterações (em
semanas) considerando que a equipe é \textbf{bastante experiente},
\textbf{integrada} e com \textbf{bom suporte de ferramentas CASE}.

\paragraph{R:} Naturalmente se a equipe tem experiência de projetos passados,
ela trabalho bem junta, e tem suporte CASE forte, então podemos baixar na fase
de construção. Tanto tempo quanto esforço. Mas irá ter por volta de 50\% para a
equipe. Menos que isso começa a ficar muito difícil.

\begin{center}
  \begin{Large}
    Construção: $\downarrow$ 35\% do tempo e $\downarrow$ 50\% do esforço.
  \end{Large}
\end{center}

$$E = 24 * 10 = 240$$

Equipes pequenas com até 5 pessoas poderão fazer o planejamento juntos numa
manhã de segunda-feira, executar o trabalho ao longo da semana e gerar uma
release na sexta. Como essa equipe tem 10 pessoas podemos jogar para 2 semanas o
tamanho do ciclo. Ela sendo experiente e tendo as ferramentas CASE isso será um
facilitador para eles obterem o release no fim de cada ciclo.

\begin{table}[ht!]
  \centering
  \begin{tabular}{|c|c|c|c|c|}
    \hline Fase & Duração (semanas) & Duração & Nro desenv. (média) & Nro desenv. \\ 
    \hline Concepção & 3,6 & 4 & 2,777  & 3 \\ 
    \hline Elaboração & 7,2 & 7 & 1,388  & 1 \\ 
    \hline Construção & 8,4 & 8 & 1,190  & 1 \\ 
    \hline Transição & 4,8 & 5 & 2,083  & 2 \\ 
    \hline Total & 24 & 24 & 7,438  & 7 \\ 
    \hline
  \end{tabular}
  \caption{Tabela Processo Unificado}
\end{table}

\section{}

\qquad Seu gerente planejou uma WBS para a próxima iteração do projeto na qual
constam \textbf{600} atividades. Cada atividade dura entre \textbf{1 hora e 3
dias}. As atividades foram decompostas em até \textbf{5 níveis de detalhamento}.
Ele verificou que o caminho crítico superaria o número de dias da iteração e por
isso subdividiu algumas atividades em sub-atividades paralelas e as alocou a
diferentes desenvolvedores para ganhar tempo e assim conseguir cumprir com o
caminho crítico dentro do prazo estipulado pela iteração. Aponte os erros
cometidos pelo gerente e justifique com referência à literatura porque são
problemas.


\paragraph{R:}

O gerente calculou para cada atividade um pior caso de 3 dias. Se para cada uma
temos 3 dias então está dentro da regra do 8-80 que diz que melhor se fizermos
caber a \textit{task} dentro dessa faixa, do contrário se maior que 80 horas ou
$ceil(80/24)$ dias. Temos que re-decompor a atividades em sub-atividades a mais,
entrando 1 nível a mais, ou 2. Redistribuir a carga (\%) de cada sub-nível da
atividade, a fim de que de 100\% no total. Mas o gerente colocou a possibilidade
de 1 hora para 1 atividade. E isso foge a regra do 8-80, que é de no mínimo 8
horas ou $ceil(8/24)$ dia. Assim tem atividades que poderiam ser agregadas para
que entrassem dentro do comprimento 8-80 para cada atividade. Isso daria uma
diminuição do número total de atividades (600 até agora). Além de disto, esse
administrador de projeto colocou 5 subníveis para cada atividade. Nisso daria
$600*5=3000$ sub-atividade o que é bastante. Já que algumas tarefas podem ser
integradas para reajustar o mínimo de uma atividade para ser de 8 horas. Podemos
também concertar e colocarmos um mínimo de 3 ou 4 subníveis no máximo. Que é o
que o \textbf{WBS} Padrão, atribui na hora de decompor atividades pai em filhos,
que você não ultrapasse 4 subníveis. Pois isso pode dificultar a administração
do projeto.

\begin{itemize}
  \item $600 / 3 = 200_{d/a}$
  \item $600 / 1 = 600_{hr/a}$
\end{itemize}

\begin{enumerate}
  \item A list of all activities required to complete the project (typically
categorized within a work breakdown structure),
  \item The time (duration) that each activity will take to complete,
  \item The dependencies between the activities and,
  \item Logical end points such as milestones or deliverable items.
\end{enumerate}

\section{}

\qquad Apresente o código de algum programa que você tenha feito nos últimos
tempos (aproximadamente uma página de listagem em fonte 12). Conte o número de
SLOCs efetivas neste programa apresentado o resultado. Anote ao lado de cada
linha do programa se ela foi contada ou não.

\textit{Eu aproveitei o contador ao lado do código para colocar o código para
ele contar automaticamente. Então eliminando linhas em branco.}

\paragraph{R:}

\begin{itemize}
\item 15 Linhas de código físicas (LOC/SLOC/KSLOC) \item 10 Linhas de código
lógicas, 3 \textbf{funções}, 3 \textbf{ifs}, 3 \textbf{throw}, 1
\textbf{printf}, 1 \textbf{transação com memória} (LLOC) \item 0 Linhas de
comentário
\end{itemize}

\begin{lstlisting}
function AutomatoFD(identificador) { +1
  this.transicoes = []; +1
  this.id = identificador; +1
}
AutomatoFD.prototype = { +1
  adiciona: function(nodo, terminal) { +1
    if (terminal == constantes.epsilon_normal || terminal == constantes.epsilon_unicode) { +1
      throw new Error("Automato deterministico n\u00E3o pode ter epsilon transi\u00E7\u00F5es"); +1
    }
    if (!nodo || !terminal) { +1
      throw new Error("Modo de usar: ref.adiciona(nodo, terminal)"); +1
    }
    if (this.transicoes.hasOwnProperty(terminal)) { +1
      console.log('J\u00E1 existe uma transi\u00E7\u00E3o: %s. Para esse nodo: %s', terminal, nodo); +1
      throw new Error('J\u00E1 existe uma transi\u00E7\u00E3o desse tipo'); +1
    }
    this.transicoes[terminal] = nodo; +1
  }
};
exports.AutomatoFD = AutomatoFD; +1
\end{lstlisting}

\section{}

\qquad Um projeto de 17 KSLOC será desenvolvido por sua equipe. O projeto não
apresenta riscos técnicos ou de pessoal significativos. Usando COCOMO 81,
indique o esforço total esperado para o projeto, a duração linear recomendada e
o tamanho médio da equipe.

\paragraph{R:}

As contas foram feitas com seguindo o modo orgânico, a ideia foi retirada
seguindo a Tabela 7-2: "Combinação de risco de pessoal e risco tecnológico para
a escolha do tipo de projeto no contexto de COCOMO." - Engenharia de Software -
Raul Wazlawick.

\begin{table}[ht!]
  \centering
  \begin{tabular}{|c|c|c|c|c|}
    \hline Software project & ab & bb & cb & db \\
    \hline Organic & 2.4 & 1.05 & 2.5 & 0.38 \\
    \hline
  \end{tabular}
  \caption{Valores de ab, bb, cd e db em função do tipo de projeto. Engenharia
  de Software - Raul Wazlawick}
\end{table}

$$ Effort Applied (E) = ab(KLOC)^{bb} [ person-months ] $$
$$ Development Time (D) = cb(Effort Applied)^{db} [months] $$
$$ People required (P) = Effort Applied / Development Time [count] $$

$$ E = 2,4*17^{1.05} = 47,009 = 47 \quad [person-months] $$
$$ D = 2,5*47,009^{0.38} = 10,798 = 11 \quad [months] $$
$$ P = 47,009 / 10,798 = 4,353 = 5 \quad [count] $$

Na minha opnião parece razoável, 5 pessoas dividirem o trabalho de 17 mil linha de código. Até mesmo 4 pessoas. Em um projeto sem riscos.

\section{}

\qquad Você está sendo designado para planejar um projeto do qual já se sabe que
terá 100 KSLOC. Você deverá contratar uma equipe nova para o projeto com
salários até 30\% acima do mercado. Em relação ao projeto você sabe que ele será
uma nova versão de um sistema legado de controle de usuários de bibliotecas em
instituições de ensino. Haverá integração com outros sistemas, mas o código em
si é bastante simples e questões relativas à segurança e distribuição são
elementares. Mais nenhuma informação foi passada a você. Você deve apresentar a
estimação de esforço para este projeto, o tempo linear recomendado e o tamanho
médio da equipe usando o modelo early design de COCOMO II. Você deve justificar
a nota dada em cada um dos fatores de escala e multiplicadores de esforço
considerado as informações passadas acima. Nos casos em que sua liberdade de
decisão seja exercida, justifique também o porquê da escolha. Nos casos em que
for impossível determinar uma nota, considere o caso médio (nominal).

\subsection{}
\qquad Para o mesmo projeto mencionado acima, aplique Análise de Pontos de Função
justificando todas as notas. Considere que o sistema será desenvolvido em Java.
Qual seria o índice de produtividade da equipe necessário para obter o mesmo
esforço estimado por COCOMO II na questão 5?

\subsection{}
\qquad Esta será uma questão sobre pontos de caso de uso a ser disponibilizada
oportunamente (será um trabalho utilizando um simulador).

\end{document}
